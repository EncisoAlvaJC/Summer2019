\documentclass[10pt,letterpaper]{article}
\usepackage[utf8]{inputenc}
\usepackage{amsmath}
\usepackage{amsfonts}
\usepackage{amssymb}
\usepackage{graphicx}
\usepackage{nicefrac}

\title{Some excercises on matrices}
\author{}
\date{}

\begin{document}

\begin{enumerate}
\item A sausage factory uses two machines for food processing:
% each one with differnt characteristics.
the machine A can process 3 Tonne (2 T) of meat in 1 hour, while the machine B can process 5 T of meat on the same time.
The goal is to process a total of 11 Tonnes on a lapse of 3 hours; to better coordinate the supply chain, both machines should end their processing at the same time.
How much meat should process each machine?

%\textit{Note:} Suppose the processing times are proportional to the total processed meat, i.e. machine A will process $\nicefrac{1}{2}$ T of meat in 1 hour, $\nicefrac{1}{4}$ T in $\nicefrac{1}{2}$ hour and so on.

\item Find a plane in $\mathbb{R}^3$ whose graph runs trough the points (4,0,-1), (-2,-3,3), (5,4,-4).

\textit{Hint:} The equation of a plane in $\mathbb{R}^3$ can be written as:
\begin{equation*}
A x + B y + Cz = 1
\end{equation*}

x|
\end{enumerate}

\end{document}